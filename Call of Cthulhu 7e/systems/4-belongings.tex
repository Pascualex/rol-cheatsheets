\section{Pertenencias y equipamiento}

Las pertenencias y el patrimonio de un personaje pueden ser un recurso útil. Interactúan principalmente con el nivel de crédito, representado como una habilidad.

\subsection{Gestión del dinero}

No es necesario hacer un seguimiento exhaustivo de los gastos. Solo los desembolsos incoherentes con los recursos del personaje deberán ser justificados.

\subsection{Nivel de vida}

No es posible hacer una tirada de crédito, en su lugar define un nivel de gastos y un patrimonio.

\vspace{-\parskip}

\begin{center}
\begin{tabular}{|c|c|}
    \hline
    Crédito & Nivel de vida     \\
    \hline
    0       & Indigente         \\
    1-9     & Pobre             \\
    10-49   & Clase media       \\
    50-89   & Adinerado         \\
    90-98   & Rico              \\
    99      & Inmensamente rico \\
    \hline
\end{tabular}
\end{center}

El nivel de gastos determina cuánto puede gastar el personaje en el día a día sin tener que justificar el pago.

El patrimonio refleja el valor de las propiedades.

\subsection{Equipamiento}

El equipamiento de un personaje no tiene reglas especiales más allá de la coherencia. Los objetos voluminosos serán visibles y los muy pesados podrán requerir tiradas de FUE.