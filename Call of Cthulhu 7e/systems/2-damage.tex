\section{Daño y curación}

El daño y la curación afectan a la salud de tu personaje, lo que puede implicar su muerte. Interactúan principalmente con los puntos de vida (PV).

\subsection{Consecuencias de la pérdida de PV}

Perder la mitad o más de los PV máximos por un único motivo provoca una herida grave, desplomarse y tirar CON para evitar perder el conocimiento.

Perder la totalidad o más de los PV máximos por un único motivo provoca la muerte directa.

Llegar a 0 PV provoca perder el conocimiento. Además, si el personaje está herido grave, pasa a estar moribundo.

\subsection{Primeros auxilios}

Tirar primeros auxilios sobre una herida provocada en la última hora recupera 1 PV.

\subsection{Tratamiento médico}

Tirar medicina sobre una herida provocada en el último día (o más antiguas con tirada difícil) recupera 1D3 PV.

\subsection{Personajes moribundos}

Un personaje moribundo no estabilizado no puede recibir tratamiento médico y tira CON cada ronda para no morir.

Recibir primeros auxilios estabiliza al personaje.

Un personaje moribundo estabilizado tira CON cada hora para evitar desestabilizarse y perder el PV recuperado.

Recibir tratamiento médico salva al personaje.

\subsection{Recuperación de PV}

Se recupera 1 PV al día mientras no se esté herido grave.

\subsection{Recuperación de una herida grave}

Un herido grave tira CON cada semana para recuperar 1D3 PV, sanando con la mitad de sus PV máximos.

Obtener un éxito extremo sana la herida al momento y hace recuperar 2D3 PV. Una pifia provoca una complicación y posiblemente una lesión permanente.

Se obtienen dados de bonificación por mantener reposo en un ambiente cómodo y por recibir tratamiento médico.

Se obtienen dados de penalización por no poder descansar y por obtener una pifia en el tratamiento médico.