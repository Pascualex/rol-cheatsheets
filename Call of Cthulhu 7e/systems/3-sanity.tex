\section{Cordura}

La cordura refleja la salud mental de tu personaje, lo que puede incluir su locura permanente. Interactúa principalmente con los puntos de cordura (COR).

\subsection{Consecuencias de la pérdida de puntos de COR}

Perder 5 o más puntos de COR por un único motivo provoca tirar INT para procesar el suceso. Si no se bloquea lo sucedido, se enloquece durante 1D10 horas. 

Perder un quinto o más de los puntos de COR del personaje en un solo día provoca enloquecer indefinidamente.

Llegar a 0 puntos de COR provoca la locura permanente, dejando de poder ser un personaje jugador.

\subsection{Tiradas de COR}

Presenciar el horror provoca tirar COR para minimizar la pérdida de puntos de COR. Una pifia maximiza la pérdida.

Fallar una tirada de COR provoca la pérdida de autocontrol unos instantes y alguna acción involuntaria.

\subsection{Primera fase de la locura: episodio de locura}

Un episodio de locura puede desarrollarse en tiempo real, perdiendo el control del personaje durante 1D10 asaltos. También puede ser resumido, con mayor duración y posiblemente sin recordar lo sucedido.

No se pueden perder puntos de COR durante un episodio.

Al finalizar, algún aspecto del personaje habrá cambia-do para siempre: apariencia, comportamiento, creencias, relaciones personales, pertenencias, fobias, manías, etc.

Si ha sido provocado por los mitos añade 5 puntos de Mitos de Cthulhu la primera ocasión y 1 las siguientes, reduciendo los puntos de COR máximos.

\subsection{Segunda fase de la locura: locura subyacente}

La locura subyacente dura hasta recuperar la cordura.

Durante este tiempo queda en manos del jugador si su personaje manifiesta externamente la locura, pudiendo actuar de manera normal y racional.

En este frágil estado, cualquier pérdida de puntos de COR desencadenará un nuevo episodio de locura.

\subsection{Fobias y manías}

Un personaje cuerdo es capaz de sobreponerse a sus fobias y manías, aunque en algunos casos pueden llegar a conllevar penalizaciones en las tiradas.

Bajo los efectos de la locura, las fobias y manías tendrán un impacto mucho mayor. Se podrá tirar psicoanálisis sobre un personaje para mitigar este efecto temporalmente.

\subsection{Tiradas de toma de conciencia}

Se podrá tirar COR para ver a través de una posible ilusión o alucinación, perdiendo 1 punto de COR si se falla.

En el caso de las alucinaciones, se podrá tirar psicoanálisis.

\subsection{Recuperación de puntos de COR}

Logros notables, como finalizar con éxito una aventura, pueden ir acompañados de una tirada de recuperación.

Un tratamiento intenso permite tirar psicoanálisis cada mes para recuperar 1D3 puntos de COR, perdiendo 1D6 puntos y finalizando el tratamiento en caso de fallo.

\subsection{Recuperación de la locura indefinida}

El mejor tratamiento es en el hogar u otro entorno familiar.

Cada mes se tira 1D100. Un 95 o menos permitirá recuperar 1D3 puntos de COR y tirar COR para superar la locura. Una fallo hará perder 1D6 puntos de COR.

El internamiento es barato pero mucho menos eficaz.